\documentclass[8pt]{./common/cheatsheet}

\cheatsheettitle{WinDbg Cheatsheet 2026 (Userland)}

\begin{document}

\begin{multicols*}{3}

\cheatsheetsection{Navigation}

\begin{tabularx}{\columnwidth}{  l  X  }
\hline
\textbf{t (F11)} & Step Into \\
\textbf{p (F10)} & Step Over \\
\textbf{g (F5)} & Run \\
\textbf{gu} & Execute until the current function is complete \\
\textbf{gn} & Execute passing exceptions to debugged process \\
\textbf{restart} & Stop and restart execution \\ 
\hline
\end{tabularx}

\cheatsheetsection{Breakpoints}

\begin{tabularx}{\columnwidth}{  l  X  }
\hline
\textbf{bl} & Lists breakpoints \\
\textbf{bp [addr]} & Set breakpoint \\
\textbf{bc \#} & Clear breakpoint (takes wildcards) \\
\textbf{bd \#} & Disable breakpoint (takes wildcards)\\
\textbf{be \#} & Enable breakpoint (takes wildcards)\\ 
\textbf{ba [a] [s] [addr]} & Hardware breakpoint where [a] is access (rwx) and [s] is size.\\ 
\hline
\end{tabularx}

\cheatsheetsection{Registers}

\begin{tabularx}{\columnwidth}{  l  X  }
\hline
\textbf{r} & Display all registers and their values \\
\textbf{r [register]} & Displays information about a single register \\
\textbf{r [register]=[value]} & Changes value of a register \\
\hline
\end{tabularx}

\cheatsheetsection{Inspecting Memory}

\begin{tabularx}{\columnwidth}{  l  X  }
\hline
\textbf{db [addr]} & displays data as Bytes + ASCII (at address [addr]) \\
\textbf{da [addr]} & displays data ASCII string until NULL is found \\
\textbf{du [addr]} & displays data UNICODE string until NULL is found \\
\hline
\end{tabularx}

\cheatsheetsection{Patching Memory}

\begin{tabularx}{\columnwidth}{  l  X  }
\hline
\textbf{eb [addr] [value]} & patches [addr] with single Byte \\
\textbf{ew [addr] [value]} & patches [addr] with word (16 bits) \\
\textbf{ed [addr] [value]} & patches [addr] with double word (32 bits) \\
\textbf{eza [addr] [value]} & patches [addr] with ASCII string \\
\textbf{ezu [addr] [value]} & patches [addr] with UNICODE string \\
\hline
\end{tabularx}

\cheatsheetsection{Searching Memory}

\begin{tabularx}{\columnwidth}{  l  X  }
\hline
\textbf{s -[type] [range] [pattern]} & search to a specific type of pattern in a memory range. \\
\textbf{s -a 0 100 "string"} & search ASCII string from address 0 until 100. \\
\hline
\end{tabularx}

\cheatsheetsection{Vieweing Memory Maps}

\begin{tabularx}{\columnwidth}{  l  X  }
\hline
\textbf{!address} & Displays all memory maps and properties. \\
\textbf{!address [addr]} & Checks if [addr] is part of a valid memory map. \\
\textbf{!address -summary} & Displays general information about memory usage. \\
\textbf{!address [filters]} & Shows only memory maps with specific properties (/f:Type=MEM\_PRIVATE and /f:Protect=PAGE\_EXECUTE\_READWRITE). \\
\textbf{!vprot [addr]} & Displays information about protection. \\

\hline
\end{tabularx}

\cheatsheetsection{Inspecting Modules}

\begin{tabularx}{\columnwidth}{  l  X  }
\hline
\textbf{lm} & Lists modules \\
\textbf{lm o} & Lists only loaded modules \\
\textbf{lm a [addr]} & Lists the module that contains the address [addr] \\
\hline
\end{tabularx}

\cheatsheetsection{Disassembler}

\begin{tabularx}{\columnwidth}{  l  X  }
\hline
\textbf{u [addr] \#} & Disassembly a number of instructions from a memory address \\
\hline
\end{tabularx}

\cheatsheetsection{Symbols}

\begin{tabularx}{\columnwidth}{  l  X  }
\hline
\textbf{ld *} & Downloads, caches and loads symbols of all loaded modules from configured source. \\
\textbf{ld [module]} & Downloads, caches and loads symbols of a specific loaded module. \\
\textbf{x module!symbol} & Display the symbols that match the specified pattern, can contain wildcard. \\
\hline
\end{tabularx}

\cheatsheetsection{Data Types}

\begin{tabularx}{\columnwidth}{  l  X  }
\hline
\textbf{dt name addr} & Specify the address of the struct (e.g. ``dt ntdll!\_TEB @\$teb''). \\
\textbf{dt -r name} & Recursively dump the subtype fields. \\
\textbf{dt name field} & Specify the specific field to display. \\
\textbf{dds [range]} & Display DWORD (4 byte) values and symbols. \\
\hline
\end{tabularx}

\cheatsheetsection{Stack}

\begin{tabularx}{\columnwidth}{  l  X  }
\hline
\textbf{k} & Display basic call stack. \\
\textbf{kp} & Display call stack with full parameters. \\
\textbf{kb} & Display call stack with three first parameters. \\
\textbf{!stack} & Summary of the current thread\'s stack usage. \\
\hline
\end{tabularx}

\cheatsheetsection{Pseudo-Registers}

\begin{tabularx}{\columnwidth}{  l  X  }
\hline
\textbf{\$peb} & Address of the current process' Process Execution Block (e.g. ``\textit{dt \_PEB @\$peb}''). \\
\textbf{\$ted} & Address of the current thread's Thread Execution Block. \\
\textbf{\$exentry} & The address of the executable's entry point. \\
\hline
\end{tabularx}

\cheatsheetsection{Help}

\begin{tabularx}{\columnwidth}{  l  X  }
\hline
\textbf{?} & Help on Debugee commands. \\
\textbf{.help} & Help on Debugger commands. \\
\hline
\end{tabularx}

\cheatsheetsection{MISC}

\begin{tabularx}{\columnwidth}{  l  X  }
\hline
\textbf{.writemem [file] [addr] L[size]} & Dumps L Bytes from address addr to file. \\
\textbf{.effmach [arch]} & Switches architecture used by the engine (interesting to analyse WoW). \\
\textbf{!exchain} & Display the current exception handler chain. \\
\textbf{!teb} & Displays the Thread Environment Block. \\
\textbf{!gle} & Shows the GetLastError code for the current thread. \\
\textbf{!error [code]} & Decodes a specific hex error code (like $0x80070005$) into its Windows error name. \\
\textbf{.printf [fmt] [prams]} & meta-command implementing C-like printf function. \\
\textbf{.reload} & Forces the debugger to discard its current symbol information and reload it. \\
\textbf{.cls} & Clears the command output screen. \\
\textbf{.sympath [path]} & Sets where WinDbg looks for symbols (``\textit{.pdb}'' files). \\
\textbf{.dvalloc} & Allocates memory using VirtualAllocEx.\\
\textbf{f [range] [pattern]} & Fills a specified memory range with a repeating pattern (\textit{e.g.} ``\textit{f 0012ff40 L20 0}'').\\
\textbf{m [range] [addr]} & Moves memory from one address to another. (\textit{e.g.} ``\textit{m 0012ff40 L20 0012ff80}'').\\
\hline
\end{tabularx}


\cheatsheetsection{UI Shortcuts}

\begin{tabularx}{\columnwidth}{  l  X  }
\hline
\textbf{ENTER} & Repeats last command. \\
\textbf{ALT + 1} & Command window. \\
\textbf{ALT + 7} & Disassembly window. \\
\textbf{Alt + Shift + T} & Opens the Threads window. \\
\hline
\end{tabularx}


\end{multicols*}

\end{document}
